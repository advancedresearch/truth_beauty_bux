\documentclass{article}

\title{Truth and Beauty Bux: Towards Sane Uses of Cryptocurrency} 

\author{Eric Purdy \\ Your Name Here (contributions welcome!)}

\begin{document}

\maketitle

In this document, we briefly lay out a variety of schemes for using
cryptocurrency that alleviate a number of the drawbacks of all
existing cryptocurrencies. We intend to implement these schemes on top
of the upcoming Thunder Token cryptocurrency, as Thunder Token seems
to be provably secure, easy to develop on (as it is planned to be
interoperable with Ethereum) and as environmentally friendly as any
cryptocurrency can be.

\section{Introduction: What is Wrong with Cryptocurrency Today?}

First, we lay out a number of objections to cryptocurrency, and note
briefly how we think they are best addressed.

\subsection{Cryptocurrencies Encourage Speculation}

Existing cryptocurrencies have largely benefited technically savvy
people with the mathematical and programming expertise required to
take advantage of less savvy participants, all while creating little
to nothing of real value. Large-scale price swings have enriched some
and devestated others, largely with no rhyme or reason. Clearly, the
root of the price swings is that cryptocurrencies have historically
sought to separate themselves from traditional, stable fiat currencies
that have widespread adoption and a clear value proposition.

The only obvious solution to these problems is to tether
cryptocurrencies to a stable fiat currency whenever possible, which
brings us to our second point:

\subsection{Tethered Cryptocurrencies Are Bullshit}

The existing tethered cryptocurrencies rely (supposedly!) on a promise
to bank vast quantities of the fiat currency and produce it upon
demand. These promises have proven to be a lie. The market may not
care that these promises are a lie, but it is clear that the
temptation to lie about such a thing is too great to resist.

Fundamentally, the promise to bank large quantites of fiat currency
and then produce them upon demand is insane and counterproductive. It
removes capital from the traditional economy while fueling the
cryptocurrency bubble. It is much wiser to invest fiat currency
backing tethered cryptocurrency in the traditional economy through
traditional avenues of investment such as mutual funds that are
subject to traditional regulation via e.g. the SEC.

We should not create any powerful institutions that are liable to
become corrupt (this was in our estimation the failure mode of the
Tether (USDT) cryptocurrency); rather, we should create an ecosystem
of institutions that are at most as powerful as is required to do
their job, and incentive structures that encourage such necessary
institutions to be transparent, honest, and straightforward, both by
designing incentives into the software of the cryptocurrency, but more
importantly, by invoking existing legal and regulatory structures and
institutions where appropriate.

Fundamentally, we believe that Tether (USDT) failed because it tried
to do something for which the incentives to be honest were too low,
and the incentives to be dishonest were too high. A USD-backed
cryptocurrency that people can withdraw from at will requires the
backing institution to either lie to investors (morally bankrupt!) or
maintain an enormous bank account which it does not use for any
investment purpose (financially unsound!).

A much wiser approach is to create an ecosystem of USD-denominated
mutual funds that accept deposits in Truth and Beauty Bux tethers, and
which has waiting periods to withdraw the tethers. This allows the
mutual funds to invest the money in a completely traditional way (in
the traditional, legal economy, and subject to traditional legal and
regulatory structures and institutions), without worrying too much
that the vagaries of the cryptocurrency markets will force them to
liquidate their investments on an unprofitable time scale; rather,
they can use the waiting period (say, 72 hours, or 168 hours) to
liquidate their investments in a sound way. Such mutual funds could
choose to waive the waiting period when the cryptocurrency markets
were not losing their minds, in order to compete for the loyalty of
their customers by providing added convenience.

\subsection{Cryptocurrencies Facilitate Illegal Transactions}

Cryptocurrencies routinely facilitate illegal transactions. This is
their primary value proposition to date. 

We believe that routine, voluntary transparency in dealings will
enable the government to do its job more easily without curtailing
civil liberties, by creating a sort of social contract between users
of the cryptocurrency. People will know that non-transparent
transactions will be subject to higher scrutiny by the government, but
the government will still have a fairly difficult time tracking down
the real participants in non-transparent transactions. Ultimately,
such efforts will probably require state-level actors with state-level
computational budgets and state-level access to traditional financial
data. It is hoped that state-level actors will be circumspect in
enforcing laws they know to be unjust, in exchange for the trust that
users place in the network by using the cryptocurrency at all, and in
exchange for the voluntary, routine transparency that this particular
cryptocurrency encourages. Ultimately, if the state-level actors
choose to enforce unjust laws (such as the prohibition of sales of
non-addictive drugs with religious or therapeutic uses, such as
hallucinogens or MDMA), and the voluntary, routine transparency
becomes less routine, the state-level actors will have only themselves
to blame.

In practical terms, this is simply an argument for signing some
transactions with cryptographic keys registered with the government
under a particular citizen's name and tax identification information.

\subsection{Cryptocurrency Has No Use Case}

This is a commonly cited argument, but we find it to be
uncompelling. If we want a use case for cryptocurrency, we need look
no further than file-sharing protocols such as BitTorrent:

Specifically, we hope to do the following:
\begin{itemize}
\item Augment the BitTorrent protocol with extensions that make it
  possible to safeguard artist's rights to be paid for the work
  they create
\item Do so without enforcing odious Digital Rights Management
  restrictions on what computations people are allowed to run on
  their own computers
\item Solve the free-rider problem that plagues existing
  BitTorrent networks by incorporating micropayments for serving
  files
\end{itemize}

We anticipate transposing the current illegal BitTorrent networks to a
completely legal and honorable method for distributing artistic
content that does not require the odious restrictions on personal
computation that were generally considered to be necessary for Digital
Rights Management.

Specifically, we envision a version of BitTorrent that exchanges {\em
  encrypted} chunks of files. These are only usable as media files if
the owner has the decryption key of the file in question. Such keys
are themselves of course perfectly copyable and perfectly
transferable, as is any digital information, and thus are themselves
subject to appearing on BitTorrent and other such platforms. It is
hoped that by cutting out such obvious rent-seekers as scientific
publishers like Elsevier, or industry groups such as the MPAA and the
RIAA, and allowing everyone together to act as a distribution
mechanism for artistic and informational content, while funneling
payments directly to artists, journalists, and other producers of such
content, it will become morally untenable for individual programmers
to circumvent such protections, and there will be little or no
political will opposing network interference with such protocols when
they are identified. Ultimately, it seems unlikely that illegal
file-sharing can ever be stamped out without violating civil
liberties, but it seems quite straightforward to make it so that it is
primarily the domain of scoundrels and scofflaws.

How do we use content distribution as a proof-of-work? This is
actually a fairly interesting thing. We postulate that nodes
participating in the modified BitTorrent protocol will receive an
encrypted file block and a nonce called the ``challenge'' that is
associated with the blockchain nodes that they are extending. They
will then publish a new tangle node that includes a hash of {\em the
  concatenation of the content of the encrypted file block with the
  challenge}. It is therefore not super useful to counterfeit such
blockchain nodes, because anyone who has a copy of the encrypted file
block in question and copies of the relevant global tangle nodes can
check your work. Presumably large actors with financial interests in
the integrity of the blockchain would simply mirror vast swathes of
encrypted content and check all nodes in the global tangle.


\subsection{Cryptocurrency Does Not Take Advantage of Existing Financial Infrastructure}

This is a valid criticism. Fundamentally, it makes more sense for
existing large financial institutions to be considered untrustworthy
but potentially valuable allies rather than enemies or competitors.

We propose to allow some blockchain verification to be done off the
global blockchain as a proof-of-work; this enables the blockchain
protocol to scale almost indefinitely, cuts down on storage needs, and
also helps to address environmental concerns.

We note that a tangle-based cryptocurrency can be made more efficient
by hosting the majority of nodes off of the global tangle. The idea
here is that network actors who have built up some level of trust can
choose to publish digests of large subtangles at regular
intervals. I.e., a large traditional financial institution such as
Goldman Sachs or the NYSE could build their own tangle, based off a
large number of nodes in the global tangle, extend with a large number
of transactions desired by their customers, then publish some number
of tips to the global tangle, along with pointers to a
cryptographically signed unencrypted content block that is available
to all network participants via the modified BitTorrent protocol. The
content block contains the tangle nodes of the transactions that
e.g. Goldman Sachs is verifying for its customers, and can be
independently verified by other institutions and by network
participants who choose to run such a computation. This can act as an
auxiliary proof-of-work, while ensuring public trust in the integrity
of the blockchain.

\section{The Automated Market-Making Problem}

We envision using an untethered host cryptocurrency (probably Thunder
Toekn) to power Truth and Beauty Bux. This necesitates conversions
between the fiat currency we are tethering to (presumably USD) and the
host cryptocurrency. Such situations have generally been taken
advantage of by savvy actors such as large financial institutions, in
a way that is essentially rent-seeking. (That is, providing liquidity
is a useful service, worthy of compensation, but in most situations,
more value has been extracted by market makers than was truly required
in order for them to make a healthy profit while serving their
customers honorably.)

We envision a distributed market-maker, in which everyone who has a
Truth and Beauty Bux wallet containing a minimum amount of both the
host cryptocurrency and Truth and Beauty Bux, and whose wallet has had
the minimum amount for more than some minimum length of time (say, one
week) will provide liquidity to the market at an automatically
determined rate. (For the sake of a very simple example protocol, say
that every blockchain node provides a bid and ask price that the
publisher of the tip must buy and sell at, and everyone else buys and
sells at the average of prices quoted on nodes that have been verified
in the past minute, or five minutes, or maybe a Gaussian-smoothed
average over the past hour (this would discourage quick price
swings).)

The effect of all this should be that people who leave amounts of
Truth and Beauty Bux above the required minimum receive compound
interest on their wallet amounts that are deposited directly into
their wallets. Thus, everyone participating in the network who
provides liquidity receives fair compensation for the service they
provide, and no network participant can take unfair advantage of their
mathematical expertise or greater market knowledge to cheat the other
network participants out of their rightful due. This seems superior in
every respect to the current situation in the cryptocurrency markets.

\subsection{Theoretical Desiderata}

In this section, we lay out a research program whereby passive market
participants (who are assumed to be less informed and less savvy) can
get a fair share of the returns that have traditionally accrued to
hedge funds and other financially savvy actors. Our general project
will be to try to construct a market-making game where the Nash
equilibrium is Pareto-optimal, and where there is no incentive for
uninformed market participants to take an active role. If we can
describe such a system and prove that it has the requisite properties,
this means that we will have achieved our goal of automated
market-making for the masses.

\section{Acknowledgments}

We thank Dr. Jesse Raber for valuable discussions related to the
intersection of the blockchain and public policy. We thank Austin
Stone for many, many valuable discussions related to the market-making
problem.

\end{document}
