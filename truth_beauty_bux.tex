\documentclass{article}

\title{Truth and Beauty Bux: Towards a Sane Cryptocurrency} 

\author{Eric Purdy \\ Your Name Here (contributions welcome!)}

\begin{document}

\maketitle

In this document, we briefly lay out a scheme for creating a new
cryptocurrency that does not suffer from the drawbacks of all existing
cryptocurrencies.

\section{Introduction}

Design goals for our cryptocurrency:
\begin{itemize}
\item Do not encourage speculation; rather, tether the currency to a
  traditional fiat currency, preferably the US dollar.
\item Do not remove value from the traditional economy; rather,
  encourage investment in the traditional economy through traditional
  avenues of investment such as mutual funds that are subject to
  traditional regulation via e.g. the SEC
\item Do not create any powerful institutions that are liable to
  become corrupt (this was in our estimation the failure mode of the
  Tether (USDT) cryptocurrency); rather, create an ecosystem of
  institutions that are at most as powerful as is required to do their
  job, and incentive structures that encourage such necessary
  institutions to be transparent, honest, and straightforward, both by
  designing incentives into the software of the cryptocurrency, but
  more importantly, by invoking existing legal and regulatory
  structures and institutions
\item Encourage routine transparency in dealings so that people can
  choose to make it easier for the government to combat illegal
  transactions and money laundering, without giving up their right to
  privacy when they desire it
\item Use a proof-of-work that serves a real human need, specifically
  content distribution and artists rights management:
  \begin{itemize}
    \item Augment the BitTorrent protocol with extensions that make it
      possible to safeguard artist's rights to be paid for the work
      they create
    \item Do so without enforcing odious Digital Rights Management
      restrictions on what computations people are allowed to run on
      their own computers
    \item Solve the free-rider problem that plagues existing
      BitTorrent networks by incorporating micropayments for serving
      files
  \end{itemize}
\item Use a tangle-based protocol in order to address both scaling
  issues and environmental concerns
\item Allow some blockchain verification to be done off the global
  tangle as a proof-of-work; this enables the blockchain protocol to
  scale almost indefinitely, cuts down on storage needs, and also
  helps to address environmental concerns
\item Address all potential concerns with the protocol using rigorous,
  testable computer science to validate all theoretical assumptions
  that underlie the security of the cryptocurrency
\item To the extent that an untethered host cryptocurrency such as
  Ethereum or IOTA is necessary to power the fiat-tethered
  cryptocurrency that we envision, encourage a healthy, automated
  market-making system that rewards all network participants who
  provide liquidity, rather than rewarding savvy individuals who know
  how to model other participants and take advantage of them.
\end{itemize}

\section{What Platform Should We Use?}

Our current thinking is to build a very lousy prototype of Truth and
Beauty Bux on top of the Ethereum blockchain, in order to validate the
market potential of the protocol on a mature and well-tested platform,
before moving to a tangle-based blockchain in order to maximize
throughput and minimize environmental externalities. This decision is
subject to revision, though, since e.g. the IOTA blockchain is
relatively mature.

A cursory glance at the IOTA blockchain suggests that tangle-based
cryptocurrencies are not thought to be capable of enforcing accurate
timestamps without some additional footwork, thus potentially
rendering them inappropriate for smart contracts. This seems
irrelevant to the problem of Truth and Beauty Bux, however, as we do
not anticipate needing to have trusted, perfect timestamps for any of
our use cases. It is a limitation to keep in mind, however.

\section{How Do We Implement An Honest Tether?}

Fundamentally, we believe that Tether (USDT) failed because it tried
to do something for which the incentives to be honest were too low,
and the incentives to be dishonest were too high. A USD-backed
cryptocurrency that people can withdraw from at will requires the
backing institution to either lie to investors (morally bankrupt!) or
maintain an enormous bank account which it does not use for any
investment purpose (financially unsound!).

A much wiser approach is to create an ecosystem of USD-denominated
mutual funds that accept deposits in Truth and Beauty Bux tethers, and
which has waiting periods to withdraw the tethers. This allows the
mutual funds to invest the money in a completely traditional way (in
the traditional, legal economy, and subject to traditional legal and
regulatory structures and institutions), without worrying too much
that the vagaries of the cryptocurrency markets will force them to
liquidate their investments on an unprofitable time scale; rather,
they can use the waiting period (say, 72 hours, or 168 hours) to
liquidate their investments in a sound way. Such mutual funds could
choose to waive the waiting period when the cryptocurrency markets
were not losing their minds, in order to compete for the loyalty of
their customers by providing added convenience.

\section{Transparency}

We believe that routine, voluntary transparency in dealings will
enable the government to do its job more easily without curtailing
civil liberties, by creating a sort of social contract between users
of the cryptocurrency. People will know that non-transparent
transactions will be subject to higher scrutiny by the government, but
the government will still have a fairly difficult time tracking down
the real participants in non-transparent transactions. Ultimately,
such efforts will probably require state-level actors with state-level
computational budgets and state-level access to traditional financial
data. It is hoped that state-level actors will be circumspect in
enforcing laws they know to be unjust, in exchange for the trust that
users place in the network by using the cryptocurrency at all, and in
exchange for the voluntary, routine transparency that this particular
cryptocurrency encourages. Ultimately, if the state-level actors
choose to enforce unjust laws (such as the prohibition of sales of
non-addictive drugs with religious or therapeutic uses, such as
hallucinogens or MDMA; or (arguably?) the prohibition of currency
transfers out of China), and the voluntary, routine transparency
becomes less routine, the state-level actors will have only themselves
to blame.

\section{The Proof of Useful Work}

We anticipate two forms of proof of work. The first is distributing
artistic and informational content. The second is verifying the
integrity of tangle fragments that are not published to the global
tangle.

Ultimately, it probably makes more sense to use a proof-of-stake based
tangle, since that is generally considered to be superior from an
environmental and scaling perspective.

\subsection{Distributing Aristic and Informational Content}

We anticipate transposing the current illegal BitTorrent networks to a
completely legal and honorable method for distributing artistic
content that does not require the odious restrictions on personal
computation that were generally considered to be necessary for Digital
Rights Management.

Specifically, we envision a version of BitTorrent that exchanges {\em
  encrypted} chunks of files. These are only usable as media files if
the owner has the decryption key of the file in question. Such keys
are themselves of course perfectly copyable and perfectly
transferable, as is any digital information, and thus are themselves
subject to appearing on BitTorrent and other such platforms. It is
hoped that by cutting out such obvious rent-seekers as scientific
publishers like Elsevier, or industry groups such as the MPAA and the
RIAA, and allowing everyone together to act as a distribution
mechanism for artistic and informational content, while funneling
payments directly to artists, journalists, and other producers of such
content, it will become morally untenable for individual programmers
to circumvent such protections, and there will be little or no
political will opposing network interference with such protocols when
they are identified. Ultimately, it seems unlikely that illegal
file-sharing can ever be stamped out without violating civil
liberties, but it seems quite straightforward to make it so that it is
primarily the domain of scoundrels and scofflaws.

How do we use content distribution as a proof-of-work? This is
actually a fairly interesting thing. We postulate that nodes
participating in the modified BitTorrent protocol will receive an
encrypted file block and a nonce called the ``challenge'' that is
associated with the tangle nodes that they are extending. They will
then publish a new tangle node that includes a hash of {\em the
  concatenation of the content of the encrypted file block with the
  challenge}. It is therefore not super useful to counterfeit such
tangle nodes, because anyone who has a copy of the encrypted file
block in question and copies of the relevant global tangle nodes can
check your work. Presumably large actors with financial interests in
the integrity of the blockchain would simply mirror vast swathes of
encrypted content and check all nodes in the global tangle.

\subsection{Verifying the Integrity of the Blockchain}

Finally, we note that a tangle-based cryptocurrency can be made more
efficient by hosting the majority of nodes off of the global
tangle. The idea here is that network actors who have built up some
level of trust can choose to publish digests of large subtangles at
regular intervals. I.e., a large traditional financial institution
such as Goldman Sachs or the NYSE could build their own tangle, based
off a large number of nodes in the global tangle, extend with a large
number of transactions desired by their customers, then publish some
number of tips to the global tangle, along with pointers to a
cryptographically signed unencrypted content block that is available
to all network participants via the modified BitTorrent protocol. The
content block contains the tangle nodes of the transactions that
e.g. Goldman Sachs is verifying for its customers, and can be
independently verified by other institutions and by network
participants who choose to run such a computation. This can act as an
auxiliary proof-of-work, while ensuring public trust in the integrity
of the blockchain.

\section{Automated Market-Making}

We envision using an existing untethered host cryptocurrency such as
Ethereum or IOTA to power Truth and Beauty Bux. This necesitates
conversions between the fiat currency we are tethering to (presumably
USD) and the host cryptocurrency. Such situations have generally been
taken advantage of by savvy actors such as large financial
institutions, in a way that is essentially rent-seeking. (That is,
providing liquidity is a useful service, worthy of compensation, but
in most situations, more value has been extracted by market makers
than was truly required in order for them to make a healthy profit
while serving their customers honorably.)

We envision a distributed market-maker, in which everyone who has a
Truth and Beauty Bux wallet containing a minimum amount of both the
host cryptocurrency and Truth and Beauty Bux, and whose wallet has had
the minimum amount for more than some minimum length of time (say, one
week) will provide liquidity to the market at a globally agreed-upon
rate. (Say, every tip provides a bid and ask price that the publisher
of the tip must buy and sell at, and everyone else buys and sells at
the average of prices quoted on tips that have been verified in the
past minute, or five minutes, or maybe a Gaussian-smoothed average
over the past hour (this would discourage quick price swings).)

The effect of all this should be that people who leave amounts of
Truth and Beauty Bux above the required minimum receive compound
interest on their wallet amounts that are deposited directly into
their wallets. Thus, everyone participating in the network who
provides liquidity receives fair compensation for the service they
provide, and no network participant can take unfair advantage of their
mathematical expertise or greater market knowledge to cheat the other
network participants out of their rightful due. This seems superior in
every respect to the current situation in the cryptocurrency markets.

\section{Acknowledgments}

We thank Dr. Jesse Raber for valuable discussions related to the
intersection of the blockchain and public policy. We thank Austin
Stone for many, many valuable discussions related to the market-making
problem.

\end{document}
